\section{Porównanie dwupętlowego regulatora PI z analitycznym regulatorem MPCS}
\indent W poniższych podrozdziałach zostaną zaprezentowane podobieństwa i różnice pomiędzy analitycznym regulatorem MPCS i dwupętlowym PI. Wynika z nich, że jeśli dysponuje się małą mocą obliczeniową lepszy rozwiązaniem będzie prosty PI. Jeśli ma się do dyspozycji bardziej wydajny sprzęt i jest taka potrzeba podczas produkcji wskazane jest użycie regulatora predykcyjnego.
\subsection{Podobieństwa}
\indent Dla obu zmiana zadanej wartości wyjścia sprawia, że na drugim wyjściu pojawia się uchyb.
\indent Oba regulatory są w stanie odpowiednio ustawić sterowania, by osiągnąć wartości zadane.
\indent Oba są podatne na błędy linearyzacji.
\subsection{Różnice}
\begin{itemize}
    \item Regulator PI działa zdecydowanie szybciej i wymaga mniejszej mocy obliczeniowej
    \item Regulator PI ma generalnie większe wartości uchybów na wyjściu drugim wyjściu po zmianie wartości zadanej na pierwszym wyjściu.
    \item Sygnały sterujące dla regulatora predykcyjnego są mniej rozmyte, tzn regulator szybciej dobiera odpowiednią wartość sterowania dla wartości zadanej.
\end{itemize}