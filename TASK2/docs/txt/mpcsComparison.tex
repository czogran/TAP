\section{Porównanie numerycznej i analitycznej wersji MPCS}
\subsection{Podobieństwa}
\begin{itemize}
    \item Dla niskich wartości horyzontów predykcji zbiegają dosyć wolno do wartości zadanej
    \item Wysoka złożoność obliczeniowa
    \item Dobre działanie dla odpowiednio dobranych parametrów
    \item Brak sprzężenia zwrotnego, podatność na błędy linearyzacji
    \item W ogólności powinny działać lepiej niż PI
\end{itemize}
\subsection{Różnice}
\begin{itemize}
    \item Wersja numeryczna ma niewielkie przeregulowania i niewielkie oscylacje dla niskich wartości horyzontów, analityczna mocno dla nich oscyluje
    \item Wersja analityczna działa nieco lepiej dla niskich opóźnień, a wersja numeryczna nieco lepiej dla wysokich opóźnień
    \item Pomimo iż oba są wymagające obliczeniowo, wersja numeryczna wymaga znacznie większej ilości obliczeń niż wersja analityczna
\end{itemize}